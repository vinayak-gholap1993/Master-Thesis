\begin{center}
\Huge \bf Tasks
\end{center}

\begin{enumerate}
\item[Task 1] Magneto-static magnetic scalar potential (MSP) formulation:
\begin{enumerate}
\item Literature survey on non-linear magneto-elasticity and its finite element modelling \cite{dorfmann2004, dorfmann2005, reddy_toroid, barham}
\item Devise a method of modelling the thin tubular membrane toroidal geometry using the MSP formulation 
\item Set up simulation code for the 3D and axisymmetric (2.5D) problem
\item Validate the axisymmetric formulation 
\end{enumerate}
\item[Task 2] Quasi-static finite strain elasticity problem:
\begin{enumerate}
\item Understand and implement a quasi-static finite-strain elasticity problem for pure mechanical loads and deformations \cite{Pelteret2016a, Pelteret2012}
\item Extend the application code from Task 1 for the axisymmetric formulation of the MSP problem to a coupled magneto-elastic problem known as a vector-valued problem
\item Implement a constitutive material law considering the geometric and material non-linearity of an isotropic continuum elastic body \cite{Wriggers2008}
\item Implement an iterative Newton method with (quasi-static linear) load incrementing algorithm to solve the non-linear system of equations for the vector-valued displacement solution
\item Test the developed code with mechanical test problems that exhibit finite deformations with instability/buckling characteristics
\end{enumerate}
\item[Task 3] Coupled magneto-elasticity problem:
\begin{enumerate}
\item Implement a material model for the coupled problem \cite{pelteret2016, Saxena2015}
\item Extend the application code incorporating the conclusions drawn from Task 1 and Task 2
\item Implement a segregated iterative solver from the literature \cite{Benzi2005} to solve the coupled saddle point system
\item Examine material behaviour at high mechanical and magnetic loads for material instability
\end{enumerate}
\item[Task 4] Path-following solution methods:
\begin{enumerate}
\item Literature survey of the path-following non-linear solvers \cite{Vasios, Wriggers2008, Riks1979, CRISFIELD1981}
\item Implement the Arc-Length solver considering the Crisfield method
\end{enumerate}
\end{enumerate}